\chapter{Právní aspekty} 
V této kapitole budou rozebrány právní aspekty výsledné aplikace, aby mohlo být rozhodnuto o možnosti zveřejnění do sítě Internet. 

\section{Zdroje dat}
Při analyzování právních aspektů aplikace je v první řadě nutné zhodnotit veškeré využité zdroje dat v aplikaci. 

\subsection{LDraw}
Aby byla dodržena licence knihovny LDraw předstevené v kapitole \ref{reserse-ldraw} při sdílení 3D modelů součástek, je nutné spolu se staženými soubory jednoznačně uvést autora. Toho je docíleno přidáním souboru LICENSE.txt do sdíleného archivu. Příklad tohoto souboru je možné vidět v ukázce kódu \ref{ukazka-licence} zobrazující licenční soubor zahrnutý v archivu stavebnice \textit{2000416-1 Duck}.

\begin{listing}[htbp]
        \begin{minted}[bgcolor=codebg]{text}
All stl files in this archive were converted by LDView from 
official LDraw Library http://www.ldraw.org/

Files are licensed under the Creative Commons - Attribution license.
http://creativecommons.org/licenses/by/2.0/

Attribution:

3021 - "Plate 2 x 3" by James Jessiman
3001 - "Brick 2 x 4" by James Jessiman
3003 - "Brick 2 x 2" by James Jessiman
3004 - "Brick 1 x 2" by James Jessiman
        \end{minted}
    \caption{Ukázka souboru LICENSE.txt\label{ukazka-licence}}
\end{listing}

\subsection{Rebrickable}
Jak již bylo řečeno v kapitole \ref{reserse-rebrickable}, Rebrickable poskytuje svá data volně bez nároků a omezení.

\subsection{Brickset}
API služby Brickset je volně dostupné k vytváření doplňkových webových stránek nebo služeb bez uvedení podmínek. \autocite{brickset:key} 

\section{0chranná známka LEGO}
Ochranná známka LEGO může být použita v případě dodržení určitých podmínek i na neoficiálních webových stránkách, na kterých je zobrazovány nebo diskutovány prvky LEGO. Instrukce vydané společností LEGO jsou dostupné na \autocite{lego:fair-play}.

Shrnutí instrukcí: 
\begin{itemize}
        \item logo LEGO nesmí být použito,
        \item ochranná známka LEGO by se měla při každém použití vždy zobrazovat symbolem ®,
        \item musí být zřejmé, že stránka není sponzorována nebo autorizována společností LEGO,
        \item ochranná známka nesmí být zvýrazňována,
        \item značka LEGO nemůže být použita v internetové adrese,
        \item stránka by měla obsahovat zřeknutí se odpovědnosti společnosti LEGO.
\end{itemize}

Všechny tyto instrukce byly zohledněny při tvrobě stránek, včetně zahrnutí zřeknutí se odpovědnosti společnosti lego do patičky (viz. ukázka kódu \ref{ukazka-prohlaseni}).

\begin{listing}[htbp]
        \begin{minted}[bgcolor=codebg]{text}
LEGO®, the LEGO logo, the Minifigure, and the Brick and Knob 
configurations are trademarks of the LEGO Group of Companies which
does not sponsor, authorize, or endorse this site.
        \end{minted}
    \caption{Zřeknutí se odpovědnosti v patičce stránky\label{ukazka-prohlaseni}}
\end{listing}

\section{Závěr}
Při implementaci byly zohledněny licenční podmínky všech služeb, využitých balíčků i instrukce společnosti LEGO pro neoficiální webové stránky. Výsledná webová aplikace tedy může být zveřejněna do sítě Internet.