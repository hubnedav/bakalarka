\chapter{Instalační příručka}\label{append:instalace}
Tato instalační příručka je dostupná také v anglické verzi v podobě README.md souboru v přiloženém zdrojovém kódu aplikace.

\section*{PrintABrick}\label{printabrick}

Webový katalog LEGO® dílů k~3D tisku

\subsubsection*{Systémové požadavky}\label{system-requirements}

\begin{itemize}
\tightlist
\item
  PHP verze 7.0 nebo vyšší
\item
  PHP rozšíření

  \begin{itemize}
  \tightlist
  \item
    FTP
  \item
    SOAP
  \item
    GD
  \item
    PDO
  \item
    Zip
  \end{itemize}
\item
  \emph{date.timezone} nastaveno v~souboru \emph{php.ini}
\end{itemize}

Zda váš systém splňuje minimální požadavky Symfony můžete zjistit příkazem \\\mintinline{bash}{$ bin/symfony_requirements}

Pro kompletní požadavky navštivte Symfony 3.3 dokumentaci \autocite{symfony:requirements}.

\paragraph{Závislosti}

\begin{itemize}
\item
  Elasticsearch \textgreater{}= 5 \autocite{elasticsearch}
\item
  POV-Ray \autocite{povray}
\item
  stl2pov \autocite{stl2pov}
\item
  ADMesh
\item
  LDView OSMesa \textgreater{}= 4.2.1 \autocite{ldview}
\end{itemize}

\subsubsection*{Instalace}\label{installing}

\paragraph{Back-end}\label{back-end}

\begin{enumerate}
\def\labelenumi{\arabic{enumi}.}
\tightlist
\item
  Ujistěte se, že váš systém splňuje minimální požadavky
\item
  Nainstalujte závislosti pomocí \href{https://getcomposer.org/}{Composer}, \mintinline{bash}{$ composer install}
\end{enumerate}

\paragraph{Front-end}\label{front-end}

\begin{enumerate}
\def\labelenumi{\arabic{enumi}.}
\tightlist
\item
  Nainstalujte npm \autocite{npm} závislosti pomocí \mintinline{bash}{$ npm install}
\item
  Nainstalujte bower \autocite{bower} závislosti pomocí \mintinline{bash}{$ bower install}
\item
  Zkompilujte zdroje \mintinline{bash}{$ gulp default [--env production]}
\end{enumerate}

\paragraph{Inicializace}\label{initialization}

\subsubsection*{Nastavení databáze}\label{setup-database}

\begin{enumerate}
\def\labelenumi{\arabic{enumi}.}
\tightlist
\item
  Nastavte parametry aplikace v~souboru \\ \mintinline{bash}{app/config/parameters.yml}
\item
  Vygenerujte prázdnou databázi příkazem \\ \mintinline{bash}{$ bin/console doctrine:database:create}
\item
  Vytvořte tabulky databáze přikazem \\ \mintinline{bash}{$ bin/console doctrine:schema:create}
\item
  Načtěte výchozí data do databáze \\ \mintinline{bash}{$ bin/console doctrine:fixtures:load}
\end{enumerate}

\subsubsection*{Načtení dat}\label{load-data}

Kompletní inicializace dat aplikace může být provedena příkazem \mintinline{bash}{$ bin/console app:init}

Tento příkaz se skládá z~několika podpříkazů, které mohou být spuštěny samostatně:

\begin{enumerate}
\def\labelenumi{\arabic{enumi}.}
\tightlist
\item
Načtení součástek knihovny LDraw \\ \mintinline{bash}{$ bin/console app:load:ldraw [--ldraw=PATH] [--all] [--file=FILE] [--update]}
\item
Načtení databáze Rebrickable \\ \mintinline{bash}{$ bin/console app:load:rebrickable} 
\item
Spárování součástek knihovny LDraw a databáze Rebrickable \\ \mintinline{bash}{$ bin/console app:load:relations} 
\item
Stažení/vykreslení obrázků součástek \\ \mintinline{bash}{$ bin/console app:load:images [--color=INT] [--rebrickable] [--missing]}
\item
Zaindexování Elasticsearch \mintinline{bash}{$ bin/console fos:elastica:populate}
\end{enumerate}


\subsubsection*{Párování součástek}\label{adding-part-relation}

Vazby mezi součástkami z~knihovny LDraw a součástkami z~databáze Rebrickable jsou automaticky spárováný při běhu příkazu \\ \mintinline{bash}{$ bin/console app:load:relations}.

Vazby mezi součástkami, které nebyly automaticky rozpoznány je možné opravit přidáním odpovídajících dvojic ID do souboru \\ \mintinline{bash}{app/Resources/relations/part_model.yml}

\subsection*{Testování}\label{testing}

Testy využívají zvláštní databázi, která musí být před prvním spuštěním testů vytvořena. Vytvoření testovací databáze:

\begin{enumerate}
\def\labelenumi{\arabic{enumi}.}
\tightlist
\item
  Vygenerujte prázdnou databázi příkazem \\ \mintinline{bash}{$ bin/console doctrine:database:create --env=test}
\item
  Vytvořte tabulky databáze přikazem \\ \mintinline{bash}{$ bin/console doctrine:schema:create --env=test}
\end{enumerate}

Kompletní systémové testy mohou být spučtěny příkazem \mintinline{bash}{$ phpunit}.

Tyto testy pokrývají základní funkcionalitu aplikace, včetně testování úspěšnosti volání programů třetích stran,využívaných aplikací.