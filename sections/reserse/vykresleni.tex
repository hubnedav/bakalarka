\section{Vykreslení modelů}\label{reserse-vykresleni}
Pro možnost poskytnout uživatelsky přívětivé a přehledné prostředí aplikace je nepochybně nutné mít možnost zobrazit součástky v~podobě statického obrázku. Obrázek většiny součástek z~knihovny LDraw je volně k~dispozici ke stažení z~webových stránek již představené služby Rebrickable \autocite{rebrickable:download}. Pro zbytek součástek je však nutné obrázek vyrenderovat. 

Nejjednodušší možností pro vyrenderování obrázků součástek je využití již představeného programu LDView, který bude v~aplikaci využíván. Program umožňuje jako jednu ze svých funkcí i vykreslení aktuálního pohledu. Tato možnost se však ukázala jako velmi nespolehlivá (v~závislosti na operačním systému, na kterém je program volán).

Proto jsem se rozhodl využít jiné nástroje k~vyrenderování obrázků součástek a to sice programy stl2pov \autocite{stl2pov} a POV-Ray \autocite{povray}.

\subsection{POV-Ray}
Dle \autocite{root:povray} je POV-Ray nástroj pro vykreslování trojrozměrných scén s~důrazem na co nejvyšší kvalitu výsledku. Celá scéna je popsána v~textovém souboru, který svou podobou připomíná jazyk C. POV-Ray takto popsanou scénu přečte, provede vykreslení a výsledek uloží ve formě rastrového obrázku. Podobu příkazu pro vykreslení scény na pozadí je možné vidět v~ukázce kódu \emph{\ref{priklad-povray}}. 

\begin{listing}[htbp]
        \begin{minted}{console}
$ povray +W900 +H600 +FN -D +I"input.pov" +O"output.png" 
        \end{minted}
    \caption{Příklad použití programu POV-Ray \label{priklad-povray}}
\end{listing}

\subsection{Stl2pov}
Stl2pov je program umožňující převod \textit{\gls{STL}} souboru na POV-Ray objekt, který je možné použít do specifikace POV-Ray scény k~vykreslení.  
