\label{introduction}
Výroba součástek a různých objektů pomocí 3D tiskáren v~domácích podmínkách se v~posledních letech stává stále rozšířenější. Je tomu tak především kvůli stále se zvyšující dostupnosti tiskáren, které si může koncový uživatel pořídit a na kterých si může tisknout své objekty. 

Mezi jedno z~možných využití tiskáren patří výroba dílů oblíbené stavebnice LEGO. Na internetu existují komunity zabývající se tvorbou 3D modelů LEGO dílů. Tyto modely jsou primárně využívány k~tvorbě virtuálních stavebnic, při jejichž tvorbě uživatel není omezován vlastnictvím určitého počtu dílů a prostorem, který stavebnice zabírá \autocite{ldraw:homepage}. Pokud se však někdo rozhodne tyto součástky převést do reálného světa a postavit si skutečnou stavebnici, existuje postup, pomocí kterého se tyto modely dají převést na vhodný formát a následně vytisknout pomocí 3D tiskárny.

Výstup práce pomůže uživatelům, kteří si chtějí vytisknout svou vlastní stavebnici. Dosavadní postupy pro získání podkladů pro tisk jsou zdlouhavé a je nutné je provést pro každou součástku stavebnice zvlášť.

Z~výše uvedených důvodů jsem se rozhodl pro volbu tématu, které výběr i tisk vlastních stavebnic usnadní a ušetří tím uživateli velké množství času jinak stráveného vyhledáváním podkladů k~tisku. 

\section*{Cíl práce}
Cílem práce je navrhnout a implementovat webovou aplikaci umožnující stažení modelů LEGO součástek a stavebnic pro 3D tisk. Návrh je potřeba provést na základě analýzy knihovny LDraw, poskytující modely jednotlivých LEGO součástek, a databáze stavebnic Brickset, které bude aplikace využívat.

Aplikace musí umožňovat vyhledávání, procházení a prohlížení samotných dílů i celých stavebnic. Součástky musí být možné prohlížet interaktivně ve 3D náhledu. Dále aplikace musí zajišťovat automatický převod dílů z~formátu knihovny LDraw do formátu vhodného pro 3D tisk a nabídnout jejich stažení jak jednotlivě, tak v~sadách podle stavebnic a barev. 

V~neposlední řadě je nutné analyzovat právní aspekty výsledné aplikace a v~případě, že to zjištěné skutečnosti dovolí, aplikaci otevřít do sítě Internet.
