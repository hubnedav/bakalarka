\chapter{Testování a nasazení}

\section{Testování}
V této sekci jsou popsány způsoby, jakými byla aplikace testována. K testování bylo využito jak metodik statické analýzy kvality kódu, tak i automatické testování pomocí jednotkových a integračních testů.

Statická analýza kvality kódu byla provedena nástrojem PHPSpec \autocite{phpspec}. Testování pomocí jednotkových a integračních testů bylo provedeno nástrojem PHPUnit \autocite{phpunit}.
Výstup testování pomocí PHPUnit se nachází v ukázce kódu \emph{\ref{phpunit-volani}}. 


\begin{listing}[htbp]
        \begin{minted}{console}
$ php vendor/bin/phpunit 
PHPUnit 6.2.2 by Sebastian Bergmann and contributors.

..................................................  50 / 121 ( 41%)
.................................................. 100 / 121 ( 82%)
.....................                              121 / 121 (100%)

Time: 41.4 seconds, Memory: 102.00MB

OK (121 tests, 191 assertions)
        \end{minted}
    \caption{Výstup PHPUnit testů \label{phpunit-volani}}
\end{listing}

\subsection{Jednotkové testy}
Jednotkové testy jsou zaměřeny na dílčí části aplikace. Měly by být maximálně izolované od ostatních částí. Hlavní podstatou jednotkových testů je kontrola jedné komponenty, bez ohledu na funkčnost ostatních komponent, u kterých předpokládáme, že fungují korektně. Z tohoto důvodu se při jednotkovém testování ostatní komponenty nahrazují falešnými objekty. \autocite{zdrojak:testy}

Implementovaný jednotkový test příkazu pro načtení knihovny LDraw je možné vidět v ukázce kódu \emph{\ref{phpunit-jednotkovy}}.

\begin{listing}[htbp]
        \begin{minted}{php}
public function testLoadAll()
{
    self::bootKernel();
    $application = new Application(self::$kernel);

    $modelLoader = $this->createMock(ModelLoader::class);
    $modelLoader->expects($this->once())->method('loadAll');

    $application->add(new LoadLdrawCommand(null,$modelLoader));

    $command = $application->find('app:load:ldraw');

    $tester = new CommandTester($command);
    $tester->execute(
        ['--ldraw' => 'path', '--all' => true]
    );
}
        \end{minted}
    \caption{Příklad jednotkového testu příkazu načtení knihovny LDraw\label{phpunit-jednotkovy}}
\end{listing}
\subsection{Integrační testy}
Integrační testy kontrolují, jak spolu třídy nebo moduly navzájem spolupracují. Testují, zda jeden modul splňuje požadavky jiného. Na rozdíl od jednotkových testů mohou využívat databázi a  souborový systém. \autocite{zdrojak:testy}

Veškeré implementované integrační testy aplikace rozšiřují třídu \textit{BaseTest}, která připravuje prostředí pro běh integračních testů. Aby bylo možné integrační testy provést, jsou vytvořena testovací data. V Symfony je k tomuto účelu určena komponenta DoctrineFixturesBundle \autocite{doctrine:fixtures}. 

Pomocí integračních testů je testována veškerá business logika aplikace, včetně testů volání programů třetích stran, využívaných k práci se soubory \textit{STL}.

\begin{listing}[htbp]
        \begin{minted}{php}
 public function testConvertToStl()
    {
        $adapter = new Local(__DIR__.'/fixtures/ldraw');
        $ldrawLibraryContext = new Filesystem($adapter);
        $this->stlConverter->setLDrawLibraryContext($ldrawLibraryContext);

        $this->assertNotNull($this->stlConverter->datToStl(__DIR__.'/fixtures/ldraw/parts/983.dat'));
        
        // Check if stl file exists
        $this->assertTrue($this->filesystem->has('models/983.stl'));
    }
         \end{minted}
    \caption{Příklad integračního testu \label{phpunit-integracni}}
\end{listing}

\subsection{Pokrytí kódu testy}
Na obrázku \emph{\ref{phpunit-coverage}} je možné vidět pokrytí kódu testy, vygenerované pomocí PHPUnit.

\begin{figure}[htbp]
    \centering
    \includegraphics[width=\textwidth,height=\textheight,keepaspectratio]{images/coverage.png}
    \caption{Pokrytí kódu testy\label{phpunit-coverage}}
\end{figure}


\section{Nasazení}
Aplikace byla nasazena na školní \gls{VPS}, který byl speciálně zřízen pro potřeby mé aplikace. Na serveru bězí operační systém CentOS 7 \autocite{centos}. Vzhedem k tomu, že jsem obdržel administrátorská práva k systému, bylo možné nainstalovat všechny potřebné závislosti, kterých aplikace využívá. 

Při instalaci bylo postupováno podle instalační příručky (příloha \emph{\ref{append:instalace}}), včetně úspešného běhu testů. 

Aplikace je nasazena a připravena k~používání na adrese \url{http://printabrick.org}. 


