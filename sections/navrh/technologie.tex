\section{Technologie}
Jednou ze základních otázek při tvorbě webové aplikace je volba technologií, které budou při implementaci využity. Od volby se odvíjí jak návrh, tak implementace samotná. 

\subsection{PHP}
Pro tvorbu webové aplikace jsem se rozhodl využít programovacího jazyka \gls{PHP}. Jedním z~hlavních kritérií zohledněných při výběru technologie byla jistá předchozí zkušenost s~programovacím jazykem PHP, který patří mezi nejpoužívanější technologie pro tvorbu webových aplikací. V~současné době je v~jazyce PHP napsáno více než 82 \% webových stránek se skriptováním na straně serveru \autocite{web:statistics}. Vzhledem k~vysoké rozšířenosti je o~jazyce PHP dostupné velké množství informací, které usnadňují vývoj. 

\subsection{Framework Symfony 3}
Po provedení průzkumu mezi PHP frameworky jsem se rozhodl svou práci vyvíjet s~využitím frameworku Symfony 3. Hlavním důvodem pro výběr Symfony byla nativní podpora konzolových příkazů \autocite{symfony:console}, kvalitně zpracovaná dokumentace a využití knihovny Doctrine \gls{ORM} \autocite{symfony:doctrine}.