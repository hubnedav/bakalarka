\section{Použité technologie}\label{implementace-technologie}

\subsection{Framework Symfony}  
Symfony je sada znovupoužitelných komponent a PHP framework pro tvorbu webových projektů \autocite{symfony}. Jedná se o~jeden z~celosvětově nejpoužívanějších nástrojů pro tvorbu webových stránek \autocite{php-frameworky}.

\subsection{Komponenty}
Při implementaci aplikace byla kromě základních komponent frameworku Symfony využita celá řada dalších komponent třetích stran:

    \paragraph{KnpMenuBundle}
    integruje do Symfony PHP knihovnu KnpMenu, která usnadňuje vytváření navigací v~aplikaci. \autocite{knpmenu} 

    \paragraph{KnpPaginatorBundle}
    integruje do Symfony PHP knihovnu KnpPagerComponent, která usnadňuje stránkování kolekcí. \autocite{knppaginator}
    
    \paragraph{LiipImagineBundle}
    poskytuje nástroj pro manipulaci s~obrázky. Umožňuje vytváření miniatur, změnu velikosti obrázku a další transformace. \autocite{liipimagine}

    \paragraph{OneupFlysystemBundle}
    je abstrakce souborového systému, která umožňuje snadnou práci se soubory a jednoduchou výměnu lokálního souborového systému za vzdálený. \autocite{oneupflysytem}
   
    \paragraph{FOSElasticaBundle}
    poskytuje integraci knihovny Elastica \autocite{elastica}, zajišťující komunikaci s~vyhledávačem Elasticsearch \autocite{elasticsearch} do Symfony. \autocite{foselastica}  
   
    \paragraph{DoctrineCacheBundle} 
    umožňuje využívání různých cachovacích systémů poskytovaných knihovnou Doctrine Cache. \autocite{doctrinecache}

\subsection{Databáze}
Framework Symfony používá pro persistenci dat \gls{ORM} Doctrine 2. Dotazy na databázi se provádějí přes entity manager a třídy
repozitářů (ve kterých jsou definované složitější dotazy), které entity manager využívají. V~Doctrine 2 mají třídy anotované atributy, které se mají persistovat. Pomocí anotací jsou definované vztahy mezi třídami, datové typy sloupců databáze a mapování.

Pro vývoj a běh aplikace jsem použil MySQL, který je druhým nejpoužívanějším databázovým serverem \autocite{database-servers}. Zároveň by však vzhledem k~abstrakci poskytované knihovou Doctrine 2 bylo možné zvolit i jiný databázový server (s~nutností zohlednění při implementaci načítání CSV tabulek Rebrickable, popsané v~sekci \emph{\ref{nacitani-rebrickable}}).

\subsection{Frontend}
Pro usnadnění práce při tvorbě projektu poskytujícího uživatelské rozhraní je vhodné využití frontend frameworku. Frameworky definují vzhled a chování nejzákladnějších, často používaných prvků. 

Pro tvorbu práce jsem se rozhodoval kromě využití pro mě již známého frameworku Bootstrap 3 také nad možností využití dalších nejpoužívanějších alternativ – Foundation a Sematntic UI \autocite{web:frameworks}. Pro mou aplikaci jsem se nakonec rozhodnul využít frameworku Semantic UI. Semantic UI při definování tříd používá přirozené názvy vycházející z~angličtiny, tudíž je jeho použití velmi intuitivní a naučení používání by mělo být velmi snadné. V~ukázce kódu \emph{\ref{ukazka-semantic}} je možné vidět příklad definice mřížky se třemi sloupečky pomocí Sematnic UI.

\begin{listing}[htbp]
  \begin{minted}{html}
<div class="ui three column grid">
  <div class="row">
    <div class="column"></div>
    <div class="column"></div>
    <div class="column"></div>
  </div>
</div>
    \end{minted}
  \caption{Ukázka definice mřížky pomocí Semantic UI\label{ukazka-semantic}}
\end{listing}