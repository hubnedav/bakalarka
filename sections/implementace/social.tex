\section{Náhled pro sociální sítě}
Významným zdrojem návštěvnosti stránek mohou v~dnešní době být i sociální sítě. Aby byl sdílený odkaz na stránku dostatečně zajímavý je třeba stránky rozšířit o~patřičné meta informace. Facebook a většina sociálních sítí ke specifikování náhledu využívá protokol Open Graph představený firmou Facebook v~roce 2010 \autocite{opengraph}.

Vše co je pro vytvoření náhledu pro sociální sítě potřeba, je přidání patřičných Open Graph meta tagů do hlavičky stránky. Pro tento účel jsem vytvořil znovupoužitelné makro šablonovacího systému \textit{Twig} (ukázka kódu \emph{\ref{opengraph-meta}}). Výsledný náhled při sdílení stránky stavebnice na sociální síti Facebook je zobrazen na obrázku \emph{\ref{facebook-share}}. 

\begin{listing}[htbp]
  \begin{minted}{twig}

    <meta property="og:title" content="{{ title }}">
    <meta property="og:description" content="{{ description }}">
    <meta property="og:url" content="{{ url }}">
    <meta property="og:image" content="{{ image }}">
    <meta property="og:image:width" content="{{ imageWidth }}">
    <meta property="og:image:height" content="{{ imageHeight }}">
    <meta property="og:site_name" content="{{ name }}">

    <meta property="twitter:card" content="summary">

  \end{minted}
  \caption{\textit{Twig} makro pro přidání Open Graph meta tagů\label{opengraph-meta}}
\end{listing}

\begin{figure}[htbp]
      \centering
      \includegraphics[width=0.7\textwidth,height=\textheight,keepaspectratio]{images/fbshare.png}
      \caption{Náhled sdílení stavebnice 2000416-1 Duck\label{facebook-share}}
\end{figure}